%%%%%%%%%%%%%%%%%%%%%%%%%%%%%%%%%%%%%%%%%%%%%%%%%%%%%%%%%%%%%%%%%%%%
\section{NMPC Problem Formulations}
%%%%%%%%%%%%%%%%%%%%%%%%%%%%%%%%%%%%%%%%%%%%%%%%%%%%%%%%%%%%%%%%%%%%
%%%%%%%%%%%%%%%%%%%%%%%%%%%%%%%%%%%%%%%%%%%%%%%%%%%%%%%%%%%%%%%%%%%%
\subsection{The NMPC Problem}
We consider a nonlinear discrete-time dynamic system expressed as \cite{economic}:
\begin{equation}
	\boldsymbol{x}_{k+1}=f(\boldsymbol{x}_k,\boldsymbol{u}_k)
	\label{eq:nonlin}
\end{equation}
where $\boldsymbol{x}_k\in\mathbb{R}^{n_x}$ denotes the state variable, $\boldsymbol{u}_k\in\mathbb{R}^{n_u}$ is the control input and $f:\mathbb{R}^{n_x}\times\mathbb{R}^{n_u}\rightarrow \mathbb{R}^{n_x}$ is a continuous model function, which calculates the next state $\boldsymbol{x}_{k+1}$ from the previous state $\boldsymbol{x}_k$ and control input $\boldsymbol{u}_k$, where $k\in\mathbb{N}$.
This system will be optimized by a nolinear model predictive controller which solves the problem:
\begin{mini!}|s|[1]
	{\boldsymbol{z}_l,\boldsymbol{v}_l}{\Psi(\boldsymbol{z}_N+\sum_{l=0}^{N-1}\psi(\boldsymbol{z}_l,\boldsymbol{v}_l)}{}{(\mathcal{P}_{NMPC}):}
	\addConstraint{\boldsymbol{z}_{l+1}=f(\boldsymbol{z}_l,\boldsymbol{v}_l), \qquad l=0,\ldots,N-1}{}
	\addConstraint{\boldsymbol{z}_0=\boldsymbol{x}_k}{}
	\addConstraint{(\boldsymbol{z}_l,\boldsymbol{v}_l)\in\mathcal{Z}}{}
	\addConstraint{\boldsymbol{z}_N\in\mathcal{X}_f}{}
\end{mini!}
at each sample time.
Here $\boldsymbol{z}_l\in\mathbb{R}^{n_x}$ is the predicted state variable; $\boldsymbol{v}_l\in\mathbb{R}^{n_u}$ is the predicted control input; and $\boldsymbol{z}_n\in\mathcal{X}_f$ is the final predicted state variable restricted to the terminal region $\mathcal{X}_f\in\mathbb{R}^{n_x}$.
The stage cost is denoted by $\psi:\mathbb{R}^{n_x}\times\mathbb{R}^{n_u}\rightarrow\mathbb{R}$ and the terminal cost by $\Psi:\mathcal{X}_f\rightarrow\mathbb{R}$.
$\mathcal{Z}$ denotes the path constraints where $\mathcal{Z}=\{(\boldsymbol{z},\boldsymbol{v})|q(\boldsymbol{z},\boldsymbol{v})\leq 0\}$, where $q:\mathbb{R}^{n_x}\times\mathbb{R}^{n_u}\rightarrow\mathbb{R}^{n_q}$.
The solution to this problem is denoted as $\{\boldsymbol{x}_0^*,\ldots,\boldsymbol{z}_N^*,\boldsymbol{v}_0^*,\ldots,\boldsymbol{v}_{N-1}^*\}$
\par
The idea is that at sample time $k$, an estimate or measurement of the state $\boldsymbol{x}_k$ is obtained and the problem $\mathcal{P}_{NMPC}$ is solved,
The first part of the optimal control sequence is then the plant input such that $\boldsymbol{u}_k=\boldsymbol{v}_0^*$.
This part of the solution defines an implicit feedback law $\boldsymbol{u}_k=\kappa(\boldsymbol{x}_k)$, and the system evolves according to Equation \ref{eq:nonlin}.
At the next sample time $k+1$, when the measurement of the new state is obtained, the procedure is repeated.
Algorithm 1 summarizes the NMPC algorithm.\\
\begin{algorithm}[H]
 \caption{General NMPC algorithm.}
	\SetAlgoLined
	set $k\leftarrow 0$\\
	\While{MPC is running}{
		\begin{enumerate}
			\item Measure or estimate $x_k$
			\item Assign the initial state: set $\boldsymbol{z}_0=x_k$
			\item Solve the optimization problem $\mathcal{P}_{NMPC}$ to find $\boldsymbol{v_0^*}$.
			\item Assign the plant input $\boldsymbol{u}_k=\boldsymbol{v}_0^*$
			\item Inject $\boldsymbol{u}_k$ to the plant
			\item Set $k\leftarrow k+1$
		\end{enumerate}
		}
\end{algorithm}
%%%%%%%%%%%%%%%%%%%%%%%%%%%%%%%%%%%%%%%%%%%%%%%%%%%%%%%%%%%%%%%%%%%%
%%%%%%%%%%%%%%%%%%%%%%%%%%%%%%%%%%%%%%%%%%%%%%%%%%%%%%%%%%%%%%%%%%%%
\subsection{Ideal NMPC and Advanced-Step NMPC Framework}
To achieve optimal economic performance and good stability properties, the problem shown in $\mathcal{P}_{NMPC}$ needs to be solved instantaneously, allowing the optimal input to be injected into the process without time delay.
This is known as ideal NMPC.
\par
In reality, there will always be some time delay between obtaining the updated values of the states and injecting them into the plant.
The main cause of this delay is the time required to solve the optimization problem $\mathcal{P}_{NMPC}$.
As the process models grow, so to does the computation time.
With sufficiently large systems, this computational delay cannot be neglected.
One approach is the advanced-step NMPC (asNMPC) which is based on the following steps:
\begin{enumerate}
	\item Solve the NMPC problem at time $k$ with a predicted state value of $k+1$
	\item When the measurement $\boldsymbol{x}_{k+1}$ becomes available at time $k+1$, compute an approximation of the NLP solution using fast sensitivity methods
	\item Update $k\leftarrow k+1$, and repeat from Step 1
\end{enumerate}
Different fast sensitivity methods can be used and are discussed further in Section \ref{sec:}.
%%%%%%%%%%%%%%%%%%%%%%%%%%%%%%%%%%%%%%%%%%%%%%%%%%%%%%%%%%%%%%%%%%%%
\section{Sensitivity-Based Path-Following NMPC}
Below we outline sensitivity results and then utilize them in a path-following scheme for obtaining fast approximate solutions to the NLP.
%%%%%%%%%%%%%%%%%%%%%%%%%%%%%%%%%%%%%%%%%%%%%%%%%%%%%%%%%%%%%%%%%%%%
%%%%%%%%%%%%%%%%%%%%%%%%%%%%%%%%%%%%%%%%%%%%%%%%%%%%%%%%%%%%%%%%%%%%
\subsection{Sensitivity Properties of NLP}
The dynamic optimization problem can be written as a generic NLP problem:
\begin{mini!}|s|[1]
	{\mathcal{X}}{F(\mathcal{X},\boldsymbol{p})}{\label{eq:param_NLP}}{(\mathcal{P}_{NLP}):}
	\addConstraint{c(\mathcal{X},\boldsymbol{p})=0}{}
	\addConstraint{g(\mathcal{X},\boldsymbol{p})\leq 0}{}
\end{mini!}
where $\mathcal{X}\in\mathbb{R}^{n_\mathcal{X}}$ are the decision variables (typically the state variables and the control input) and $\boldsymbol{p}\in\mathbb{R}^{n_p}$ is the parameter (typically the initial state variable).
$F:\mathbb{R}^{n_\mathcal{X}}\times \mathbb{R}^{n_p}\rightarrow\mathbb{R}$  is the scalar objective function, $c:\mathbb{R}^{n_\mathcal{X}}\times \mathbb{R}^{n_p}\rightarrow\mathbb{R}^{n_c}$ denotes the equality constraints, and $g:\mathbb{R}^{n_\mathcal{X}}\times \mathbb{R}^{n_p}\rightarrow\mathbb{R}^{n_g}$ denotes the inequality constraints.
Each instance of the general parameteric NLP shown in Equation \ref{eq:param_NLP} that are solved for each sample time differ only in the parameter $\boldsymbol{p}$.
(See Suwardti et. al. for the Lagrangian and the Karush-Kuhn-Tucker (KKT) conditions \cite{economic}.)
\par
We wish to know how the solution changes with a perturbation in the parameter $\boldsymbol{p}$.

