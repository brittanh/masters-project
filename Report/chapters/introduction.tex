Model predictive control (MPC), along with non-linear model predictive control (NMPC), is an advanced control strategy that involves solving an optimization problem for a set horizon to determine the feedback value of the manipulated variables at each sampling interval.
Traditionally this control strategy was only widely used in the chemical industry for processes with large time constants (i.e., slow dynamics).
However, due to modern computation capabilities and algorithm development, this type of control has expanded to a variety of system types (even fast dynamics).
MPC has a growing interest in both research and industry due to its performance in a variety of processes in addition to its ability to handle constraints, perform optimization all while considering economics and nonlinearities of the process.
The current areas of interest are: development of algorithms for rapid optimization, development of better modeling strategies, and new alternatives/variations that lead to improved closed-loop performance or reduce the computation time of the optimization problem.
\par
Since most industries care most about the profitability of the process, another type of MPC/NMPC, known as economic MPC (eMPC), was developed.
This allows for the integration of the economic optimization and the control layer into a single dynamic optimization layer (\cite{economic}).
Economic MPC works by adjusting the inputs such that the economic cost of the operation is directly minimized; thus allowing for the optimization of the cost during operation of the plant.
When an optimization-based controller such as MPC is used, the economic criterion can be included directly in the cost function of the controller (\cite{fast}).
However, nonlinear process models are often used for this style of optimization meaning that a drawback of economic MPC is the requirement of solving large nonlinear optimization problem (NLP) with the NMPC problem at every sample time.
This computation can take a significant amount of time and lead to increasingly worse performance and even instability of the process (\cite{economic}).
\par
One idea to reduce the effect of computational delay in NMPC is to use sensitvity-based methods which exploit the fact that the NMPC optimization problems are identical at each sample time with the exception of one changing parameter: the initial state.
Therefore, the full nonlinear optimization problem is no longer solved.
Instead, the sensitivity of the NLP solution at the previously-computed iteration is used to obtain an approximate solution to the new NMPC problem (\cite{economic}).
One such method is the advanced-step NMPC (asNMPC) which involves solving the full NLP at every sample time but this solution is computed in advance for a predicted initial state.
When the new state measurement is available from the process, the NLP solution is corrected using a fast sensitivity update to make the solution match the measured state.
\par
In this project, we focused on applying an improved path-following method for correcting the NLP solution within the advanced-step NMPC framework in Python. 
We illustrate how asNMPC with the predictor-corrector path-following algorithm performs in the presence of measurement noise and compare it with an ideal NMPC approach, where the NLP is assumed to be solved instantly.